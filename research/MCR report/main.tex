%Mid-Year Cadidature Review Report 
%Updated on 23/08/2012 at 15.30

\documentclass{article}  
\usepackage{a4wide}
\usepackage{amsmath,amssymb}
\usepackage{graphicx}
\usepackage{url}
\usepackage{tabulary}
\usepackage[sort]{cite}
%\usepackage[sort,nocompress]{cite}
\usepackage{multirow}
\usepackage{booktabs}
\usepackage{placeins}
\usepackage{caption}
\usepackage{subcaption}
\usepackage{epstopdf}
\usepackage{enumerate}
\usepackage{pdfpages}

%\title{\Huge bfseries{Towards a green optical Internet 
%jnvsjkvnsfjkvnfkjsv 
%dfdfd
%}} 
%\author{\textsf{M. Nishan Dharmaweera}} 

%\year{April, 2004} 
%\department{\emph{Department of Electrical and Computer Systems Engineering}}
%\title{Towards a Green Optical Internet}
%\author{Nishan Dharmaweera \and Rajendran Parthiban \and Y. Ahmet \c{S}ekercio\u{g}lu}
\begin{document}

\begin{titlepage}
\begin{center}
{\huge \bfseries Energy harvesting using aeroelastic galloping}\\[2.5cm]
{\LARGE \bfseries H.G.K.G Jayatunga}\\[2.5cm]
\textsc{\Large Supervisors:\\[0.5cm] Dr. Tan Boon Thong \\[0.4cm] Dr. Justin Leontini \\[0.5cm] Dr Huang Yew Mun}\\[6.5cm]
\textsc{\Large Ph.D. Mid-Candidature Review Report}\\

\vfill
\textsc{\Large $27^{\text{th}}$ September 2013}
\end{center}
\end{titlepage}
%\titlepage
%\maketitle
\tableofcontents

\section{Introduction}
The search for alternative energy sources which could be categorised under the ”green”label has become important area of research in the modern word. Solar, wind power andwave power are some of the examples of these sources. Recently,a new branch of research has been developing to extract energy from flow induced vibrations. It has been hypothesized that this technique may work efficiently in areas where regular turbines cannot.
A simple structure that is susceptible to flow-induced vibrations that are suitable for energy extraction are slender structures,such as cylinders, elastically mounted perpendicular to a fluid stream. With regards to a slender body two common types of   flow induced vibrations are Vortex Induced Vibrations (VIV) and aeroelastic galloping. Significant research has been carried out by Bernitsas and his team on extracting useful energy from VIV. Some of their significant work includes investigating the influence  of physical parameters such as mass ratio (the ratio of the mass of the cylinder and the displaced fluid), Reynolds number, mechanical properties(\cite{Raghavan2010a} ,\cite{Lee2011b}) and location (effect of the bottom boundary) (\cite{Raghavan2009}). However,the possibility of extracting energy using aeroelastic galloping has not been thoroughly investigated. Some theoretical work was carried out by (\cite{Barrero-Gil2010a}). Utilizing galloping may be a more viable method to harness energy from flow induced vibrations as it is not bounded by a "lock-in" range of reduced velocities(ratio between the freestream velocity and the product of the natural frequency of the system and the characteristic length).Therefore it is preferable to investigate further the possibility of harnessing energy from flow induced vibrations using aeroelastic galloping. 

\section{Research focus}
 This research contributes to the existing knowledge on energy extraction from flow induced vibrations. This research focuses investigating methods to optimise the energy transfer from fluid-to-body when galloping by using theoretical methods (i.e Quasi-steady state theory and DNS simulations). Therefore the project was separated into two main phases.
 
\begin{enumerate}[]
\item Phase 1: To optimise the mechanical portion of the system 
\item Phase 2: To optimise the fluid dynamics portion of the system 
\end{enumerate}










\section{Research objectives}










\section{Research progress}

From the literature survey, we identified both optical-circuit switching (OCS) and optical-burst switching (OBS) to be more energy efficient in comparison to point-to-point (pt-pt) and optical-packet switching (OPS) technologies. However, both OCS and OBS face a number of challenges. Therefore, in stage two of this study, we attempted to generate novel mechanisms to increase energy efficiency in OCS and OBS based optical fiber backbone networks. 
\begin{itemize}
\item 
\end{itemize}


\begin{itemize}
\item 

\end{itemize}
Finally, in stage four of this study, we intend to explore and develop novel frameworks to increase energy efficiency in an optical fiber backbone network, by utilizing either sleep mode of operation, ALR or MLR techniques. As the reader might have noticed, both sleep mode of operation and MLR were used in the two previous two stages in conjunction with OBS and WBS technologies. We are currently analysing the existing literature, and are in the process of developing an energy-aware network configuration where line cards and links in the network could move back and forth between sleep and active states depending on the network load. 

With respect to the PhD thesis, each of the above stages (literature survey and the three approaches) are intended to be converted into a chapter. This is to be followed by a chapter covering the discussion and the conclusion (Appendix). During my PhD, the three main approaches are being studied concurrently. Therefore, as of today, certain tasks from each different stage have been completed. The remaining tasks are to be completed before the end of the allocated time period as explained in the Gantt chart shown in Figure 1. In conclusion, 60\% of the research work has been completed to date with 70\% of the objectives being successfully accomplished.

%\begin{figure}[h]
%\centering
%\includegraphics[width=\textwidth]{./Gantt}
%\caption{Revised Gantt Chart}
%\label{fig:Gantt}
%\end{figure}


\section{Difficulties and challenges} 
%
%\begin{enumerate}
%      
%\end{enumerate}
\bibliographystyle{elsarticle-harv}
\bibliography{../bibtex/MCR}
\appendix
%\begin{center}
%\section*{\LARGE Appendix 1}
%\addcontentsline{toc}{section}{Appendix 1} 
%\end{center}
Proposed table of contents of the final PhD thesis:
\begin{enumerate}
\item Introduction
\begin{enumerate}[i]
\item Looking forward
\item Scope of the thesis
\item Organization of the thesis
\item Objectives
\item Contributions
\item Publications 
\end{enumerate}
\item Power consumption in optical core network
\begin{enumerate}[i]
\item Introduction
\item Core network architecture
\item Core node architecture 
\item Power consumption values
\item Node-based approaches
\item Traffic engineering-based approaches
\item Network engineering-based approaches
\end{enumerate}
\item Node-based energy efficiency
\begin{enumerate}[i]
\item Introduction
\item General network model
\item Problem definition
\item Cost model
\item Energy and cost efficient algorithm
\item Framework for energy efficient OBS network
\item Summary
\end{enumerate}
\item Traffic engineering-based energy efficiency
\begin{enumerate}[i]
\item Introduction
\item Grooming based network model
\item Problem definition
\item Sparse grooming 
\item Waveband grooming 
\item Summary
\end{enumerate}
\item Network engineering-based energy efficiency \begin{enumerate}[i]
\item Introduction
\item MLR based network model
\item Problem definition
\item MLR based WBS network
\item Sleep mode enabled core network
\item Summary
\end{enumerate}
\item Conclusion
\begin{enumerate}[i]
\item Introduction
\item Summary of the work
\item Future directions
\end{enumerate}
\end{enumerate}
Each sub-section mentioned above could be further divided into sections, if necessary. For example, 3(v) could be divided as 3(v)(a) "ILP formulation", 3(v)(b) "Heuristic algorithm" and 3(v)(c) "Evaluation"
\newpage
\centering
\vspace*{2.5in}
\section*{\LARGE Appendix 2}
\addcontentsline{toc}{section}{Appendix 2} 
\includepdf[page={-}]{ICTpaper.pdf}
\end{document}