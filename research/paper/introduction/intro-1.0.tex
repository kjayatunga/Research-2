\section{Introduction} 

The search for alternate energy sources with minimal environmental impact has become an important area of research in the modern word. Solar, wind power and wave power are some of the examples of these sources.  Recently, a new branch of research has been developing to extract energy from flow induced vibrations \citep{Bernitsas2008a-concept}. It has been hypothesized that this technique may work efficiently in areas where regular turbines cannot. 

An elastically-mounted slender structure such as a cylinder which is susceptible to flow-induced vibrations has the potential for energy extraction. With regards to slender bodies, two common types of flow-induced vibrations are vortex-induced vibrations (VIV) and aeroelastic galloping. Significant research has been carried out by Bernitsas and his team on extracting useful energy from VIV. Some of their significant work includes investigating the influence  of physical parameters such as mass ratio, Reynolds number, mechanical properties \citep{Raghavan2010a, Lee2011b} and the influence of the proximity of a solid boundary \citep{Raghavan2009}. However, the possibility of extracting energy using aeroelastic galloping has not been thoroughly investigated. Some theoretical work was carried out by \citet{Barrero-Gil2010a}. Utilizing galloping may be a more viable method to harness energy from flow-induced vibrations as it is not bounded by a narrow ``lock-in'' range of reduced velocities (\ustar). This study further explores the possibility of harnessing energy from flow induced vibrations using aeroelastic galloping.

According to \citet{Paidoussis2010}, \citet{Glauert1919} provided a criterion for galloping by considering the auto-rotation of an aerofoil.  \citet{DenHartog1956} provided a theoretical explanation for galloping for iced electric transmission lines. A weakly non-linear theoretical aeroelastic model to predict the response of galloping was developed by \citet{Parkinson1964} based on the quasi-steady state (QSS) theory. Experimental lift and drag data on a fixed square prism at different angles of attack were used as an input for the theoretical model. It essentially used a curve fit of the transverse force to predict the galloping response. The study managed to achieve a good agreement with experimental data.

However, the QSS model equation when solved analytically using the sinusoidal solution method cannot predict the response for cases with low mass ratios. \citet{Joly2012} observed that finite element simulations show a sudden change in amplitude below a critical value of the mass ratio. The model equation defined in \citet{Parkinson1964} was modified to account for the vortex shedding and and solved numerically to predict the reduced amplitude at low mass ratios to the point where galloping is no longer present. \citet{Barrero-Gil2010a} investigated the possibility of extracting power from vibrations caused by galloping using the quasi-steady state model. In the conclusions of that paper it was pointed out that in order to obtain a high power to area ratio, the mass-damping ($m^*\zeta$) parameter should be kept low. The same study investigated the influence of the characteristics of the $C_y$ curve on maximum power output.

\JL{Kasun: This is not finished! How does your current study build on these results? Do your conclusions differ from those of Barrero-Gil and Joly?}

\JL{Also, you need to give a description of what is coming in the paper. Something like:

Here, the modified QSS model developed by \citet{Joly2012} is integrated numerically. The focus is on the power extraction potential as a function of mass ratio. To this end, a series of mass ratios are tested at two different values of \reynoldsnumber: $\reynoldsnumber = 165$, a case that should remain laminar and essentially two-dimensional; $\reynoldsnumber = 22300$, a case where the flow is expected to be turbulent and three-dimensional. Both cases require the input of transverse force coefficients $C_y$ as a function of angle of attack $\alpha$ for a fixed body. These data are provided from direct numerical simulations for the $\reynoldsnumber = 165$ case, while the data provided by \citet{Parkinson1964} are used for the $\reynoldsnumber = 22300$ case.

The structure of the paper is as follows. Section \ref{sec:theory} presents the modified QSS model, the method for the calculation of power output, and the parameters used. Section \ref{sec:results} presents the results, first of the fixed body tests at a range of $\alpha$, then of the response characteristics predicted by the integration of the QSS model for both the high and low \reynoldsnumber\ cases. For the low \reynoldsnumber\ case, the results of the QSS model are compared to those of full direct numerical simulations of the fluid-structure interaction problem. Finally, section \ref{sec:conc} presents the conclusions that can be drawn from this work.}