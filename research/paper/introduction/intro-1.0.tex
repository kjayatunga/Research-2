\section{Introduction} 

The search for alternate energy sources which could be categorised  under the "green" label has become important area of research in the modern word. Solar, wind power and wave power are some of the examples of these sources.  Recently,a new branch of research has been developing to extract energy from flow induced vibrations(\cite{Bernitsas2008a-concept}). It has been hypothesized that this technique may work efficiently in areas where regular turbines cannot. 

A simple structure that is susceptible to flow-induced vibrations that are suitable for energy extraction are slender structures,such as cylinders, elastically mounted perpendicular to a fluid stream. With regards to a slender body two common types of   flow induced vibrations are Vortex Induced Vibrations (VIV) and aeroelastic galloping. Significant research has been carried out by Bernitsas and his team on extracting useful energy from VIV. Some of their significant work includes investigating the influence  of physical parameters such as mass ratio (the ratio of the mass of the cylinder and the displaced fluid), Reynolds number, mechanical properties(\cite{Raghavan2010a} ,\cite{Lee2011b}) and location (effect of the bottom boundary) (\cite{Raghavan2009}). However,the possibility of extracting energy using aeroelastic galloping has not been thoroughly investigated. Some theoretical work was carried out by (\cite{Barrero-Gil2010a}). Utilizing galloping may be a more viable method to harness energy from flow induced vibrations as it is not bounded by a "lock-in" range of reduced velocities(ratio between the freestream velocity and the product of the natural frequency of the system and the characteristic length).Therefore it is preferable to investigate further the possibility of harnessing energy from flow induced vibrations using aeroelastic galloping.

Real life energy harvesting systems use high damping ratios where the energy generator (e.g electrical generator) puts a significant amount of damping into the system. Therefore it is crucial to investigate  the behaviour of aeroleastic galloping scenarios at high damping ratios in order to optimise the system to obtain a acceptable power output.Hence the focus of this paper is concentrated on investigating the mechanical power output of high-damped galloping systems in laminar flow.

According to \cite{Paidoussis2010},\cite{Glauert1919} has provided a criterion for galloping by considering the auto-rotation of an aerofoil.  \cite{DenHartog1956}  has provided a theoretical explanation for galloping for iced electric transmission lines. A non-linear theoretical aeroelastic model to predict the response of galloping was developed by \cite{Parkinson1964} based on the quasi-steady state (QSS) theory. Experimental lift and drag data on a fixed square prism at different angles of attack were used as an input for the theoretical model. It essentially used a curve fit of the transverse force to predict the galloping response. The study managed to achieve a good agreement with experimental (wind tunnel) data.\cite{Joly2012} have observed that finite element simulations shows a sudden change in amplitudes  below a critical values of the mass ratio, which the  (QSS) model fails to reproduce. The Parkinson's equation was essentially modified to account for the vortex shedding and managed to produce the effects to the amplitude at low mass ratios.
\cite{Barrero-Gil2010a} have investigated the possibility of extracting power from vibrations caused by galloping using quasi-steady state theory. In the conclusions of that paper it was pointed out that in order to obtain a high power to area ratio the mass-damping ($m^*\zeta$) parameter should be kept low as well as the frequency of oscillations should be carefully matched \hilight{have a good agreement with the size of the cross section}. Another interesting conclusion was that energy conversion systems which uses galloping could operate over  a large range of flow velocities unlike VIV energy harvesting systems where the factor of energy conversion has a strong dependence on the incoming flow velocity. 